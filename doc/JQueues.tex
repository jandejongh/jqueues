\documentclass[12pt]{book}

\usepackage{color}
\usepackage{listings}
\usepackage{array}

\title{Discrete-Event Simulation\\
       of Queueing Systems in Java:\\
       The JSimulation and JQueues Libraries}
\author{Jan de Jongh}
\date{Release 5}

\lstset
{
  language=Java,
  basicstyle=\ttfamily,
  commentstyle=\textit,
  keywordstyle=\color{blue}\bfseries,
  tabsize=2,
  frame=single
}

\begin{document}

\maketitle

\chapter*{}

{\em This book is dedicated to Dick Epema and Graham Birtwistle.}

\chapter{Preface}

Queueing systems deal with the general notion of {\em waiting\ }
  for (the completion of) something.
They are ubiquitously and often annoyingly present in our everyday lives.
If there is anything we do most,
  it is probably {\em waiting\ } for something to
  happen (finally winning a non-trivial prize in the State Lottery
          after paying monthly tickets over the past thirty years),
  arrive (the breath-taking dress we ordered from that webshop
          against warnings in the seller's reputation blog),
  change (the reception of many severely bad hands in the poker game
          we just happened to ran into),
  stop (the constant flipping into red of traffic lights
        while we are just within breaking distance
        in our urban environment),
  or resume (the heater that regularly happens to have
             a mind of its own during
             winter months).

{\bf XXX}

\chapter{Introduction}

\chapter{Guided Tour}

\chapter{Events, Event Lists and Actions}

This chapter describes the event and event-list features
  that are available from the \lstinline{jsimulation} package.
Note that \lstinline{jsimulation} is a dependency of \lstinline{jqueues}.

\section{Creating the Event List and Events}

At the very heart of every simulation experiment
  in \lstinline{jqueues}
  is the so-called {\em event list}.
The event list obviously holds the events,
  keeps them ordered,
  and maintains a notion of "where we are" in a simulation run.
Together, an event list and the events it contains define
  the precise sequence of actions taken in a simulation.
The following code snipplet shows how to create an event list and
  schedule two (empty) events, one at $t_{1}=5.0$ and one at $t_{2}=10$,
  and print the resulting event list on \lstinline{System.out}:
\begin{lstlisting}
final SimEventList el = new SimEventList ();
final SimEvent e1 = new SimEvent (5.0);
final SimEvent e2 = new SimEvent (10.0);
el.add (e1);
el.add (e2);
el.print ();
\end{lstlisting}
In \lstinline{jsimulation},
  the event list is of type \lstinline{SimEventList};
  events are of type \lstinline{SimEvent},
  respectively.
Typically,
  you instantiate a single event list for a simulation experiment,
  and numerous events.

The \lstinline{double} argument in the \lstinline{SimEvent} constructor
  (of which there are several)
  is the {\em schedule time\/} of the event on the event list.
Perhaps surprisingly,
  in \lstinline{jsimulation},
  the schedule time is actually held on the event,
 {\em not\/} on the event list.
Moreover, a \lstinline{SimEventList} is implemented as a \lstinline{TreeSet}
  from the Java Collections Framework.
These implementation choices have the following consequences:
\begin{itemize}
  \item Each \lstinline{SimEvent} can be present {\em at most once\/} in a \lstinline{SimEventList}.
        You cannot reuse a single event instance (like a job creation and arrival event)
          by scheduling it multiple times on the event list.
        Instead, you must either use separate event instances, or reschedule the event
          the moment it leaves the event list.
  \item You cannot (more precisely, {\em should not\/}) modify the time on the event while it is
          scheduled on an event list.
  \item You always have access to the (intended) schedule time of the event, without having to
          refer to an event list (if the event is scheduled at all) or use a separate
          variable to keep and maintain that time.
\end{itemize}

The output of the code snipplet is something like\footnote{
We may have improved the layout in the meantime.}:
\begin{lstlisting}[basicstyle=\tiny]
SimEventList nl.jdj.jsimulation.r4.SimEventList@159bfbe, time=-Infinity:
  time=5.0, name=No Name, object=null, action=null.
  time=10.0, name=No Name, object=null, action=null.
\end{lstlisting}
The output shows the name of the event list (as obtained from its \lstinline{toString} method)
  and the current time ($-\infty$) in the first row, and then the events in the list
  in the proper order.
The output also shows the four properties of an event: its time, name, user object, and action.
These will be described in more detail in the next section.

\section{Event Properties and Event Constructors}

A \lstinline{SimEvent} has the following properties:
\begin{itemize}
\item Time:   The (intended) schedule time of the event.
\item Name:   The name of the event, which is only used for logging and output.
\item Object: A general-purpose object available for storing information associated with the event
              (\lstinline{jsimulation} nor \lstinline{jqueues} use this field).
\item Action: The action to take, a \lstinline{SimEventAction}, described in the next section.
\end{itemize}

\section{Actions}

A \lstinline{SimEventAction} defined what needs to be done by the time an event
  is {\em executec} or {\em processed}.
In Java terms, a \lstinline{SimEventAction} is an interface with
  a single abstract method which is invoked when the event is processed.
Below we show the declaration of the interface:
\begin{lstlisting}[basicstyle=\tiny]
@FunctionalInterface
public interface SimEventAction<T>
{

  /** Invokes the action for supplied {@link SimEvent}.
   *
   * @param event The event.
   *
   * @throws IllegalArgumentException If <code>event</code> is <code>null</code>.
   * 
   */
  public void action (SimEvent<T> event);

}
\end{lstlisting}

There are several ways to create actions for events.
The first and most often used way in our own code is to use anonymous inner classes:
\begin{lstlisting}[basicstyle=\tiny]
final SimEventList el = new SimEventList ();
final SimEvent e =
  new SimEvent ("My First Real Event", 5.0, null, new SimEventAction ()
  {
    @Override
    public final void action (final SimEvent event)
    {
      System.out.println ("Event=" + event + ", time=" + event.getTime () + ".");
    }
    @Override
    public String toString ()
    {
      return "My First Action";
    }
  });
el.add (e);
el.print ();
el.run ();
el.print ();
\end{lstlisting}
Note that we are now using the full \lstinline{SimEvent} constructor,
  giving out own name, and supplying a \lstinline{SimEventAction}
  as an anonymous inner class.
In the inner class, we define the \lstinline{action} method,
  and in the meantime override the \lstinline{toString} method
  (to be honest, this was merely to keep the generated text within bounds).
The generated output is:
\begin{lstlisting}[basicstyle=\tiny]
SimEventList nl.jdj.jsimulation.r4.SimEventList@106d69c, time=-Infinity:
  t=5.0, name=My First Real Event, object=null, action=My First Action.
Event=My First Real Event, time=5.0.
SimEventList nl.jdj.jsimulation.r4.SimEventList@0, time=5.0:
  EMPTY!
\end{lstlisting}
Clearly, as expected!
However, rote that after "running" the event list, it turns out to be empty,
  and its time is now $t=5.0$, the schedule time of our event.
This is as intended, and will be explained in the next section.
But first we look at an alternative way of attaching
  actions to events:
\begin{lstlisting}[basicstyle=\tiny]
final SimEventList el = new SimEventList ()
{
  @Override
  public final String toString ()
  {
    return "My Renamed Event List";
  } 
};
final SimEventAction action = new SimEventAction ()
{
  @Override
  public final void action (final SimEvent event)
  {
      System.out.println ("Event=" + event + ", time=" + event.getTime () + ".");
  }
  @Override
  public final String toString ()
  {
    return "A Shared Action";
  }
};
for (int i = 1; i <= 10; i++)
{
  final SimEvent e = new SimEvent ("Our Event", (double) i, null, action);
  el.add (e);
}
el.print ();
el.run ();
el.print ();
\end{lstlisting}
In this example, we created a single action object
  (again using an anonymous inner class),
  and reuse it among ten distinct events we schedule
  (we cannot reuse those).
We also took the opportunity give our
  event list a friendlier name by overriding its \lstinline{toString} method.
The output is as follows:
\begin{lstlisting}[basicstyle=\tiny]
SimEventList My Renamed Event List, time=-Infinity:
  t=1.0, name=Our Event, object=null, action=A Shared Action.
  t=2.0, name=Our Event, object=null, action=A Shared Action.
  t=3.0, name=Our Event, object=null, action=A Shared Action.
  t=4.0, name=Our Event, object=null, action=A Shared Action.
  t=5.0, name=Our Event, object=null, action=A Shared Action.
  t=6.0, name=Our Event, object=null, action=A Shared Action.
  t=7.0, name=Our Event, object=null, action=A Shared Action.
  t=8.0, name=Our Event, object=null, action=A Shared Action.
  t=9.0, name=Our Event, object=null, action=A Shared Action.
  t=10.0, name=Our Event, object=null, action=A Shared Action.
Event=Our Event, time=1.0.
Event=Our Event, time=2.0.
Event=Our Event, time=3.0.
Event=Our Event, time=4.0.
Event=Our Event, time=5.0.
Event=Our Event, time=6.0.
Event=Our Event, time=7.0.
Event=Our Event, time=8.0.
Event=Our Event, time=9.0.
Event=Our Event, time=10.0.
SimEventList My Renamed Event List, time=10.0:
  EMPTY!
\end{lstlisting}
Again note that the time on the event list after running it
  is the time of the last event we scheduled on it.

So, there are different ways of attaching a \lstinline{SimEventAction}
  to a \lstinline{SimEvent}.
The abundant use of anonymous inner classes as shown here
  is certainly not to everyone's taste,
  but it results in relatively compact code
  (even more through the use of lambda expressions, see XXX).

\section{Processing the Event List}

Once the events of your liking are scheduled on the event list,
  you can start the simulation by {\em processing\/} or {\em running\/}
  the event lists.
Processing the event list will cause the event list to
  equentially invoke the actions attached to the events
  in increasing-time order.
There are several ways to process a \lstinline{SimEventList}:
\begin{itemize}
  \item You can process the event list until it is empty with the \lstinline{run} method.
  \item You can process the event list until some specified (simulation) time with the
          \lstinline{runUtil} method.
  \item You can {\em single-step\/} through the event list with the
          \lstinline{runSingleStep} method.
\end{itemize}
You can check whether an event list is being processed through its \lstinline{isRunning}
  method.

While processing, the event list maintains a {\em clock}
  holding the (simulation) time of the current event.
You can get the time from the event list through \lstinline{getTime} nethod,
  although you can obtain it more easily from the event itself.
You can insert new events while it is being processed,
  {\em but these events must not be in the past}.
Once the event list detects insertion of events in the past,
  it will throw and exception.

Note that processing the event list
  is thread-safe in the sense that all methods involved
  need to obtain a {\em lock} before being able to process the list.
Trying to process an event list that is already being processed
  from another thread,
  or from the thread that currently processes the list,
  will lead to an exception.
Note that currently there is no safe, atomic, way
  to process an event list on the condition that is
  is not being processed already.
Though you can check with \lstinline{isRunning}
  whether the list is being processed or not,
  the answer from this method has zero validity lifetime.

The example below shows how to schedule new events
  from event actions; it also shows what happens if you schedule
  events in the past.
\begin{lstlisting}[basicstyle=\tiny]
final SimEventList el = new SimEventList ()
{
  @Override
  public final String toString ()
  {
    return "The Event List";
  } 
};
final SimEventAction schedulingAction = new SimEventAction ()
{
  private int counter = 0;
  @Override
  public final void action (final SimEvent event)
  {
      System.out.println ("Event=" + event + ", time=" + event.getTime () + ".");
      counter++;
      if (counter < 10)
        // Schedule 1 second from now.
        // Use utility method on SimEventList.
        el.schedule (event.getTime () + 1, this);
      else if (counter == 10)
      {
        // Schedule now.
        el.schedule (event.getTime (), this);
        System.out.println ("Scheduled event now.");
      }
      else
      {
        // Schedule 1 second in the past -> throws exception.
        el.schedule (event.getTime () - 1, this);
        // Never reached.
        System.out.println ("Scheduled event in the past.");
      }
  }
  @Override
  public final String toString ()
  {
    return "Scheduling Action";
  }
};
el.schedule (0, schedulingAction);
el.print ();
el.run ();
el.print ();
\end{lstlisting}
The code begins to look familiar.
First, we create the event list, then a single action.
The action is a bit more complicated than before;
  it has an internal \lstinline{counter} in the anynoumous class.
Using the counter, it reschedules itself ten times,
  the first nine times one second in the future,
  the tenth time at exactly the same time.
As mentioned before, this is perfectly legal
  (and, in fact, often used in our own code).
The final attempt to reschedule the action results in an
  exception, because the event is scheduled in the past.
Note that the example also showcases a utility method
  in \lstinline{SimEventList}, viz., \lstinline{schedule (double, SimEventAction)},
  which directly schedules the action on the event list at given time,
  creating a new \lstinline{SimEvent} on the fly.
In a later section we will look in more detail at more utility methods
  on event lists.

The output of the example is shown below\footnote{
  For improved reading, we have left out the full stack-trace of the exception,
  and rearranged the mixed outputs from \lstinline{System.out} and \lstinline{System.err}.
  We will do that without notice in the sequel.
}.
\begin{lstlisting}[basicstyle=\tiny]
SimEventList The Event List, time=-Infinity:
  t=0.0, name=No Name, object=null, action=Scheduling Action.
Event=No Name, time=0.0.
Event=No Name, time=1.0.
Event=No Name, time=2.0.
Event=No Name, time=3.0.
Event=No Name, time=4.0.
Event=No Name, time=5.0.
Event=No Name, time=6.0.
Event=No Name, time=7.0.
Event=No Name, time=8.0.
Event=No Name, time=9.0.
Scheduled event now.
Event=No Name, time=9.0.
Exception in thread "main" java.lang.IllegalArgumentException:
Schedule time is in the past: 8.0 < 9.0!
\end{lstlisting}
Note that in this particular case,
  the exception thrown actually comes with an
  instructive message as to what caused it
  (you tried to schedule something on the event list at $t=8.0$,
   whereas the current time is beyond that, $t=9.0$).
However, in all honesty,
  such messages are not present
  for the majority of exceptions thrown
  as a result of incorrect arguments from user code.
We are currently working on improving this.

The output also shows the expected result from the first \lstinline{el.print} statement:
Only a single event is scheduled!
The others are created and scheduled while the event list is being processed.
It is important to realize that the contents of a \lstinline{SimEventList}
  can always change, as long as these are changes {\em now or in the future}.
By the way, the second invocation of \lstinline{el.print} does not stand a chance;
  it is unreachable because of the exception thrown in \lstinline{el.run}.

\section{Utility Methods for Scheduling Events}

A \lstinline{SimEventList} supports various methods for
  directly scheduling events and actions
  without the need to generate both
  the \lstinline{SimEvent} {\em and\/} the \lstinline{SimEventAction}.
In most cases, the availability of one of the object suffices.
Below we show the most common utility methods for scheduling on a \lstinline{SimEventList}.

\begin{tabular}{|l|}
  \hline
  {\bf Utility methods for scheduling} \\
  \hline
  \lstinline[basicstyle=\footnotesize]!void schedule (E)! \\
    Schedules the event at its own time.\\
  \hline
  \lstinline[basicstyle=\footnotesize]!void schedule (double, E)! \\
    Schedules the event at given time.\\
  \hline
  \lstinline[basicstyle=\footnotesize]!reschedule (double, E)! \\
    Reschedules (if present, else schedules) the event at given new time.\\
  \hline
  \lstinline[basicstyle=\footnotesize]!E schedule (double, SimEventAction, String)! \\
    Schedules the action at given time with given event name.\\
  \hline
  \lstinline[basicstyle=\footnotesize]!void scheduleNow (E)! \\
    Schedules the event now.\\
  \hline
  \lstinline[basicstyle=\footnotesize]!E schedule (double, SimEventAction)! \\
    Schedules the action at given time with default event name.\\
  \hline
  \lstinline[basicstyle=\footnotesize]!E scheduleNow (SimEventAction, String)! \\
    Schedules the action now with given event name.\\
  \hline
  \lstinline[basicstyle=\footnotesize]!E scheduleNow (SimEventAction)! \\
    Schedules the action now with default event name.\\
  \hline
\end{tabular}

Note that \lstinline{E} refers to the so-called {\em generic-type argument\/}
  of \lstinline{SimEventList}.
The prototype is \lstinline!SimEventList<E extends SimEvent>!.
The use of generic types is explained in some more details in the "Advanced Topics" section,
  but for now \lstinline!E! can be simply read as a \lstinline{SimEvent}.

For any of the utilty methods that take a \lstinline{SimEventAction}
  as argument, a new \lstinline{SimEvent} is created on the fly,
  and returned from the method.
Upon return from these methods,
  the newly created event has already been scheduled,
  and you {\em really\/} should not schedule it again.

You may wonder how to {\em remove\/} events and actions from the event list.
Well, since \lstinline{SimEventList} implements the \lstinline{Set} interface for
  \lstinline{SimEvent} members, removing an event \lstinline{e}
  from an event list \lstinline{el} is as simple as
  \lstinline{el.remove (e)}.
Currently, there is no support to remove an action from an event list.
Because actions can be reused, it would require iterating over
  all scheduled events,
  and remove all events with the given action.
It is not hard to implement at all, we just did not do it\footnote{
This code fragment has not been tested.}:
\begin{lstlisting}[basicstyle=\tiny]
public static void removeAction
(final SimEventList eventList, final SimAction action)
{
  if (eventList != null)
  {
    final Iterator it = eventList.iterator;
    while (it.hasNext ())
      if (it.next ().getEventAction () == action)
        it.remove ();
  }
}
\end{lstlisting}
The code fragment silently assumes
  the absence of \lstinline{null} events
  in the event list,
  which is indeed guaranteed,
  and works perfectly for \lstinline{null} actions.
Note the somewhat unexpected method name on \lstinline{SimEvent}
  to get its action, viz., \lstinline{getEventAction}.
This name was chosen in order to avoid potential name clashes.
At the risk of sounding pedantic,
  the explicit use of the iterator
  looks old-fashioned,
  yet allows for
  the safe removal of elements
  from a collection in a loop
  (contrary to a much fancier \lstinline{for} construction).

We conclude with an overview of
  non-scheduling related utility methods
  of \lstinline{SimEventList}:

\begin{tabular}{|l|l|}
  \hline
  {\bf Method} & {\bf Description} \\
  \hline
  \lstinline[basicstyle=\footnotesize]!void print ()! & Prints the event list to \lstinline!System.out!. \\
  \lstinline[basicstyle=\footnotesize]!void print (PrintStream)! & Prints the event list to the stream. \\
  \hline
\end{tabular}

\section{Simultaneous Events}

While reading through the previous sections,
  you may have wondered
  what would happen
  if two events are scheduled
  on exactly the same time.
Well, why not just give it a try?
First, we create an action class with an index number as argument;
  when invoked, the action merely prints its index number
  to \lstinline{System.out}:
\begin{lstlisting}[basicstyle=\tiny]
private static class IndexedSimEventAction
implements SimEventAction
{
  
  final int index;
  
  public IndexedSimEventAction (final int index)
  {
    this.index = index;
  }
  
  @Override
  public void action (SimEvent event)
  {
    System.out.println ("Hello, I am action number " + this.index + "!");
  }

  @Override
  public String toString ()
  {
    return "Action " + index;
  }
  
}
\end{lstlisting}
So, let us schedule some of these at $t=0$
  in order of increasing index:
\begin{lstlisting}[basicstyle=\tiny]
final SimEventList el = new SimEventList ();
for (int i = 1; i <= 10; i++)
  el.schedule (0, new IndexedSimEventAction (i), "Event " + i);
el.print ();
el.run ();
\end{lstlisting}
The potential result of this code may be a bit surprising\footnote{
The probablity of you seeing the same result is $1/(10!)$,
  which equals the probability
  of you being {\em not\/} surprised at all about your own output.}:
\begin{lstlisting}[basicstyle=\tiny]
SimEventList nl.jdj.jsimulation.r4.SimEventList@a27aa42, time=-Infinity:
  t=0.0, name=Event 9, object=null, action=Action 9.
  t=0.0, name=Event 5, object=null, action=Action 5.
  t=0.0, name=Event 8, object=null, action=Action 8.
  t=0.0, name=Event 7, object=null, action=Action 7.
  t=0.0, name=Event 6, object=null, action=Action 6.
  t=0.0, name=Event 4, object=null, action=Action 4.
  t=0.0, name=Event 3, object=null, action=Action 3.
  t=0.0, name=Event 1, object=null, action=Action 1.
  t=0.0, name=Event 10, object=null, action=Action 10.
  t=0.0, name=Event 2, object=null, action=Action 2.
Hello, I am action number 9!
Hello, I am action number 5!
Hello, I am action number 8!
Hello, I am action number 7!
Hello, I am action number 6!
Hello, I am action number 4!
Hello, I am action number 3!
Hello, I am action number 1!
Hello, I am action number 10!
Hello, I am action number 2!
\end{lstlisting}
Well, it looks like all our scheduled events were indeed processed,
  {\em but not in the order we inserted them into the list!}
It was even clear {\em before\/} processing the event list that there
  was something "wrong" with the sequence of events.
Why?
Well, because we explictly instructed the \lstinline{SimEventList}
  {\em not\/} to do process simultaneous events in so-called
  {\em insertion order},
  but instead to break ties {\em at random\/} for
  simultaneously scheduled events.
The exact reasoning for doing this is a bit involved,
  and deferred until the "Advanced Topics" section,
  but for now it is important to realize that
  a \lstinline{SimEventList}
\begin{itemize}
  \item processes its scheduled events in random order
        should they have equal schedule times;
  \item will {\em never\/} preempt or interrupt the current event it is processing
        in favor of another event that is scheduled at the same time from within the
        action of the current event;
  \item allows insertion-order processing of simultaneous events if you really want to.
\end{itemize}

\section{Resetting an Event List}

\section{Listening to an Event List}

\section{Advanced Topics}

\subsection{Using Generic-Type Arguments}

\subsection{Simultaneous Events: Random-Order and Insertion-Order Event Lists}

\subsection{Action is a Functional Interface}

\chapter{Queueing Systems; Entities, Queues, and Jobs}

\chapter{Fundamental Queues}

\section{Introduction}

\section{Serverless Queues}

\section{Single-Server Queues}

\section{Finite-Server Queues}

\section{Infinite-Server Queues}

\section{Processor-Sharing Queues}

\section{Preemptive Queues}

\chapter{Visualization of Queues and Jobs with Swing Components}

\chapter{Queue and Job Statistics}

\chapter{Multiclass Queues and Jobs}

\chapter{Advanced Topics}

\section{Job Factories}

\section{Queue Events and Schedules}

\section{Load Factories}

\chapter{Composite Queues}

\section{Introduction}

Composite queues consist of zero or more other queues
  named {\em subqueues\/} or {\em embedded queues\/}
  through which visiting jobs must pass.
The sequence of visits to the embedded queues is
  determined by the composite queue.

{\bf XXX}

\section{Types (Colors) of Composite Queues}

\section{Tandem Queues}

\section{Compressed Tandem Queues}

\section{Parallel Queues}

\section{Feedback Queues}

\section{Jackson Networks}

\section{Special Composite Queues}

\subsection{Encapsulator Queues}

\subsection{Drop-Collector Queues}

\section{Building Custom Queues and Jobs}

\section{Test Infrastructure}

\chapter{Conclusions}

\end{document}
